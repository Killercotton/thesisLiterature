wes\chapter{Chapter 1}
\label{ch:one}

\section{Read Literature}
\begin{itemize}
	\item An Evaluation of the Suitability of FPGAs for Embedded Vision Systems, 2005
	\item Embedded vision system for real-time object tracking using an asynchronous transient vision sensor, 2006
	\item Stereo Vision System on Programmable Chip (SVSoC) for Small Robot Navigation, 2006
	\item A 128x128 120dB 15$\mu$s Latency Asynchronous Temporal Contrast Vision Sensor, feb 2008
	\item A Tele-operated Gesture Recognition Mobile Robot using a Stereo Vision, 2008
	\item Smart cameras, 2010
	\item Dynamic stereo vision system for real-time tracking, 2010
	\item Accurate hardware-based stereo vision, 11.08.2010
	\item Active vision in robotic systems: A survey of recent developments, 2011
	\item PIXHAWK: A micro aerial vehicle design for autonomous flight using onboard computer vision, 23.02.2012
	\item Low-cost FPGA stereo vision system for real time disparity maps calculation, 27.02.2012
	\item Parallelization of Belief Propagation on Cell Processors for Stereo Vision, June 2012
	\item An Overview to Visual Odometry and Visual SLAM: Applications to mobile robotics, 13.11.2015
	\item Precise localisation of a UAV using visual odometry, December 2015
	\item Potential-path approach for UAV with movable camera on-board, January 2017
\end{itemize}

\section{Summaries}
\subsection{Dynamic stereo vision system for real-time tracking} \label{sec:DSV}
\begin{itemize}
\item paper describes an embedded system with two special vision sensors and a DSP
\item objective is the tracking of objects in address-even representation
\item intensity changes over time are registered as events
\item events are grouped together with a bounding box to recognize moving objects
\item recognized objects can be tracked
\item use of AER reduces necessary computation power
\item DVS very expensive
\item use the event stream coming from the sensor, rectify pixels, put them in a time frame and match corresponding pixels
\end{itemize}

\subsection{Embedded vision system for real-time object tracking using an asynchronous transient vision sensor}
\begin{itemize}
\item uses DVS for monitoring of traffic and persons tracking
\item uses event(pixel detecting intensity change) clusters to track objects
\item every new event is added to a cluster and influences cluster center
\item path of cluster is path of cluster center
\end{itemize}

\subsection{A 128x128 120dB 15$\mu$s Latency Asynchronous Temporal Contrast Vision Sensor}
\begin{itemize}
\item this paper describes a sensor that asynchronously detects intensity changes and sends the address of the corresponding pixel
\item this includes circuit design, testing, previous work
\item seems hard to replicate this exact idea on an fpga
\item this is the DVS sensor used in the above paper
\end{itemize}

\subsection{Smart cameras}
\begin{itemize}
\item a book about all kinds of image sensors, image processing systems
\item interesting chapter about embedded stereo vision
\item extensively describes algorithm in \ref{sec:DSV}
\end{itemize}

\subsection{PIXHAWK: A micro aerial vehicle design for autonomous flight using onboard computer vision}
\begin{itemize}
	\item paper extensively describes a lightweight drone with a unusually powerfull computation platform
	\item drone capable of using complex computer vision algorithms normally reserved for heavier drones
	\item improves navigation by combining inertia measurment unit with computer vision data
	\item objective is to build an autonomous drone that does all calculations online
	\item article not wholly finished as extensively describes all aspects of design
\end{itemize}

\subsection{Parallelization of Belief Propagation on Cell Processors for Stereo Vision}
\begin{itemize}
	\item paper explains complicated algorithm that minimizes an energy function to find matching pixels between the two images of a camera pair
	\item algorithm is rather complicated
	\item algorithm is very accurate but needs a lot of resources
	\item shows that the algorithm can be used on a multicore system using parallelization to achieve about 6 fps
	\item necessary resources not available on ramstix, therefore not relevant
\end{itemize}

\subsection{A Tele-operated Gesture Recognition Mobile Robot using a Stereo Vision}
\begin{itemize}
	\item bad English, not very nice to read
	\item service robot that recognizes face and hands and hand gestures
	\item does computations on board and on a server
	\item mention an FPGA but not clear if it is on the server or on the robot
	\item main topic is an algorithm for face recognition and hand gesture recognition
\end{itemize}

\subsection{Low-cost FPGA stereo vision system for real time disparity maps calculation}
\begin{itemize}
	\item describes a complete system that performs stereo vision
	\item implementation on a XlinixVirtex-4 XC4VLX60 FPGA chip
	\item consists of image acquisition, image rectification, disparity computation and validation
	\item reaches over 300 fps for a resolution of 640x480
	\item claims to perform well while costing 1/10 of comparable systems
\end{itemize}

\subsection{Accurate hardware-based stereo vision}
\begin{itemize}
	\item article about a new algorithm for real-time systems to be implemented on FPGAs or ASICs
	\item uses sum of absolute differences (SAD) in combination with a modified census transform (MCT)
	\item these two are combined to get rather high accuracy for an embedded system while reaching a framrate of 60 fps
	\item article complicated, explains used algorithm in detail
	\item write an IP core for FPGAs that can be adapted to the size of the FPGA, this way it also fits on smaller FPGAs
\end{itemize}

\subsection{An Evaluation of the Suitability of FPGAs for Embedded Vision Systems}
\begin{itemize}
	\item looks at advantages and disadvantages of using FPGAs in an embedded computer vision context
	\item FPGAs are very powerful when one can use parallelism
	\item Seems to be the case for many computer vision algorithms
	\item biggest disadvantage is the need to use fixed-point arithmetic
	\item floating-point arithmetic is very costly
	\item easy to make and change a design
\end{itemize}

\subsection{Stereo Vision System on Programmable Chip (SVSoC) for Small Robot Navigation}
\begin{itemize}
	\item describes a stereo vision system that recognizes obstacles and plans a possible path
	\item uses three cameras together with a board that combines FPGA and DSP
	\item stereo vision seems to be done only on FPGA
	\item allows a hexapod drone to avoid obstacles
\end{itemize}

\subsection{Potential-path approach for UAV with movable camera on-board}
\begin{itemize}
	\item RAM report of Elisa Rimondi about a camera that stays focused on a high-feature view to improve localisation of drones
	\item extensive with a lot of background information
	\item stereo visual odometry search term for more suitable literature
	\item explains different position systems such as GPS or odometry
	\rro{How does visual odometry work?}
	\rro{What does stereo vision add to visual odometry and when is it necessary?}
	\item SLAM is the process of generating a map and simultaneously the position of a drone in it while both depend on each other
	\rro{How important is SLAM for Aeroworks? Seems that map creation is not necessary}
	\item active vision means that a drone actively chooses a path with a lot of features to reduce localisation uncertainty
	\rro{active vision insofar relevant as it doesn't change the planned path, i.e. camera orientation changes to search high-feature areas}
	\item uses Robot Operating System (ROS) 
	\item ROS is a framework that basically provides a way of connecting different kinds of software elements as nodes
	\item it provides topics that all nodes can publish to and read messages from
	\item describes different forms of feature recognition
	\rro{What criteria are important for the choice of a feature detector?}
	\item every pixel in image gets an energy and the camera points at location with lowest energy to keep camera as stable as possible
	\item extensive description of system architecture with all ros-nodes, coordinate systems and transformations
	\item improves localisation by focussing camera on features
	\item only achieves 15Hz because of ROS latency
	\rro{How can frequency be improved?}
	\item proposes using ROS2 in combination with global shutter high framerate camera with large field of view
	\rro{Can stereo vision be done with large field of view? fish-eye lens?}
	\item would be interesting to compare Elisa's method with other methods of odometry to evaluate performance and combine localisation data with other sensor data
\end{itemize}

\subsection{An Overview to Visual Odometry and Visual SLAM: Applications to mobile robotics}
\begin{itemize}
	\item recounts history of VO and VSLAM
	\item VO: match features in different frames to compute translation matrix
	\rro{How can that be implemented?}
	\rro{How can VO data be combined with other sensor data to improve accuracy? (Kalman filter)}
	\item RANSAC often used to improve feature detection
	\rro{What is RANSAC? (see further below)}
	\item explains different kinds of SLAM
	\item explains different feature extraction algorithms and how to associate features in different frames
	\rro{What are good criteria to choose the most suitable feature extraction algorithm and data association?}
	\item RANSAC is an algorithm to find outliers in measurement data
	\item Bundle Adjustment seeks to optimize camera pose and 3D structure parameters by minimizing a cost function
\end{itemize}

\subsection{Precise localisation of a UAV using visual odometry}
\begin{itemize}
	\item aims at using stereo vision on a UAV in combination with IMU for localisation
	\item uses an AEROWORKS drone with vi-sensor
	\item fuses SVO measurements with IMU with a kalman filter
	\rro{How does a kalman filter work? (described in this paper)}
	\item makes use of ROS
	\item chooses between different implementation options for localisation and measurement fusion
	\rro{What does noise look like in VO?}
	\item uses an intel NUC i5 with Linux Ubuntu 14.04 and ROS Jade
	\item SVO is fused with IMU data to get more accurate localisation and orientation of the drone
	\item IMU has drift over time but SVO as well
	\item when SVO fails, no pose is available, therefore backup pose estimation used
	\item measurements from flight data
	\item error in some directions bigger than in others, yaw introduces error
	\item transformation between coordinate frames important
	\item motors introduce noise for IMU
	\item contains recommendations such as using more sensors or eliminating drift with SLAM
	\rro{Other ways of reducing drift in VO?}
\end{itemize}

\subsection{Active vision in robotic systems: A survey of recent developments}
\begin{itemize}
	\item gives an overview of important development within the field of Active Vision in robotics
	\item 
\end{itemize}